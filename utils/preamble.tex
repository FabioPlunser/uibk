%%%%%%%%%%%%%%%%%%%%%%%%%%%%%%%%
% PACKAGE IMPORTS
%%%%%%%%%%%%%%%%%%%%%%%%%%%%%%%%%


\usepackage[tmargin=2cm,rmargin=1in,lmargin=1in,margin=0.85in,bmargin=2cm,footskip=.2in]{geometry}
\usepackage{amsmath,amsfonts,amsthm,amssymb,mathtools}
\usepackage[varbb]{newpxmath}
\usepackage{xfrac}
\usepackage[makeroom]{cancel}
\usepackage{mathtools}
\usepackage{bookmark}
\usepackage{enumitem}
\usepackage{hyperref,theoremref}
\hypersetup{
	pdftitle={Assignment},
	pdfborder={0 0 0},
	colorlinks=false,
	bookmarksnumbered=true,
	bookmarksopen=true
}
\usepackage[most,many,breakable]{tcolorbox}
\usepackage{xcolor}
\usepackage{varwidth}
\usepackage{varwidth}
\usepackage{etoolbox}
%\usepackage{authblk}
\usepackage{nameref}
\usepackage{multicol,array}
\usepackage{tikz-cd}
\usepackage[ruled,vlined,linesnumbered]{algorithm2e}
\usepackage{comment} % enables the use of multi-line comments (\ifx \fi) 
\usepackage{import}
\usepackage{xifthen}
\usepackage{pdfpages}
\usepackage{transparent}
\usepackage{listings}
\usepackage{logicproof}
\usepackage{./utils/kvDiagram}
\usepackage{xparse}
\usepackage{svg}
\usepackage{graphicx}
\usepackage{datetime}
\newdateformat{myformat}{\THEDAY{ten }\monthname[\THEMONTH], \THEYEAR}
\usepackage{lmodern}
\usepackage{tcolorbox}
\usepackage{titlesec}
\usepackage{fancyhdr}
\usepackage{verbatim}




\newcommand\mycommfont[1]{\footnotesize\ttfamily\textcolor{blue}{#1}}
\SetCommentSty{mycommfont}
\newcommand{\incfig}[1]{%
    \def\svgwidth{\columnwidth}
    \import{./figures/}{#1.pdf_tex}
}

\usepackage{tikzsymbols}
\renewcommand\qedsymbol{$\Laughey$}

%\usepackage{import}
%\usepackage{xifthen}
%\usepackage{pdfpages}
%\usepackage{transparent}


%%%%%%%%%%%%%%%%%%%%%%%%%%%%%%
% SELF MADE COLORS
%%%%%%%%%%%%%%%%%%%%%%%%%%%%%%



% One Dark pro
\definecolor{black}{HTML}{282c34}
\definecolor{white}{HTML}{abb2bf}
\definecolor{red}{HTML}{e06c75}
\definecolor{darkRed}{HTML}{be5046}
\definecolor{green}{HTML}{98c379}
\definecolor{yellow}{HTML}{e5c07b}
\definecolor{darkYellow}{HTML}{d19a66}
\definecolor{blue}{HTML}{61afef}
\definecolor{magenta}{HTML}{c678dd}
\definecolor{cyan}{HTML}{56b6c2}
\definecolor{gutterCyan}{HTML}{4b5263}
\definecolor{commentGrey}{HTML}{5c6370}

\definecolor{myg}{HTML}{98971a}
\definecolor{myb}{HTML}{458588}
\definecolor{myr}{HTML}{cc241d}
\definecolor{myexercisebg}{HTML}{F2FBF8}
\definecolor{myexercisefg}{HTML}{88D6D1}
\definecolor{mytheorembg}{HTML}{F2F2F9}
\definecolor{mytheoremfr}{HTML}{00007B}
\definecolor{mylemmabg}{HTML}{FFFAF8}
\definecolor{mylemmafr}{HTML}{d79921}
\definecolor{mypropbg}{HTML}{f2fbfc}
\definecolor{mypropfr}{HTML}{191971}
\definecolor{myexamplebg}{HTML}{F2FBF8}
\definecolor{myexamplefr}{HTML}{689d6a}
\definecolor{myexampleti}{HTML}{8ec07c}
\definecolor{mydefinitbg}{HTML}{E5E5FF}
\definecolor{mydefinitfr}{HTML}{3F3FA3}
\definecolor{notesgreen}{HTML}{98971a}
\definecolor{myp}{HTML}{b16286}
\definecolor{mygr}{HTML}{b8bb26}
\definecolor{myred}{HTML}{fb4934}
\definecolor{myyellow}{HTML}{d79921}


%%%%%%%%%%%%%%%%%%%%%%%%%%%%
% TCOLORBOX SETUPS
%%%%%%%%%%%%%%%%%%%%%%%%%%%%

\setlength{\parindent}{1cm}
%================================
% THEOREM BOX
%================================


%================================
% Lemma
%================================
\newtcbtheorem[number within=section]{Lemma}{Lemma}{enhanced,
	before skip=2mm,after skip=2mm, colbacktitle=blue!15, colback=blue!10,colframe=blue!80!black,boxrule=0.5mm,
	attach boxed title to top left={xshift=1cm, yshift*=1mm-\tcboxedtitleheight}, varwidth boxed title*=-3cm,
	boxed title style={frame code={
					\path[fill=blue]
					([yshift=-1mm,xshift=-1mm]frame.north west)
					arc[start angle=0,end angle=180,radius=1mm]
					([yshift=-1mm,xshift=1mm]frame.north east)
					arc[start angle=180,end angle=0,radius=1mm];
					\path[left color=blue!60!black,right color=blue!60!black,
						middle color=blue!80!black]
					([xshift=-2mm]frame.north west) -- ([xshift=2mm]frame.north east)
					[rounded corners=1mm]-- ([xshift=1mm,yshift=-1mm]frame.north east)
					-- (frame.south east) -- (frame.south west)
					-- ([xshift=-1mm,yshift=-1mm]frame.north west)
					[sharp corners]-- cycle;
				},interior engine=empty,
		},
	fonttitle=\bfseries,
}
{Lemma}

%================================
% Definition Box
%================================
\newtcbtheorem[number within=section]{Definition}{Definition}{
  enhanced,
	before skip=2mm,after skip=2mm, colbacktitle=red!15, colback=red!10,colframe=red!80!black,boxrule=0.5mm,
	attach boxed title to top left={xshift=1cm, yshift*=1mm-\tcboxedtitleheight}, varwidth boxed title*=-3cm,
	boxed title style={frame code={
					\path[fill=red]
					([yshift=-1mm,xshift=-1mm]frame.north west)
					arc[start angle=0,end angle=180,radius=1mm]
					([yshift=-1mm,xshift=1mm]frame.north east)
					arc[start angle=180,end angle=0,radius=1mm];
					\path[left color=red!60!black,right color=red!60!black,
						middle color=red!80!black]
					([xshift=-2mm]frame.north west) -- ([xshift=2mm]frame.north east)
					[rounded corners=1mm]-- ([xshift=1mm,yshift=-1mm]frame.north east)
					-- (frame.south east) -- (frame.south west)
					-- ([xshift=-1mm,yshift=-1mm]frame.north west)
					[sharp corners]-- cycle;
				},interior engine=empty,
		},
	fonttitle=\bfseries,
}{Definition}

%================================
% Theorem
%================================
\newtcbtheorem[number within=section]{Theorem}{Theorem}{
  enhanced,
	before skip=2mm,after skip=2mm, colbacktitle=yellow!15, colback=yellow!10, colframe=yellow!80!black,boxrule=0.5mm,
	attach boxed title to top left={xshift=1cm, yshift*=1mm-\tcboxedtitleheight}, varwidth boxed title*=-3cm,
	boxed title style={frame code={
					\path[fill=yellow]
					([yshift=-1mm,xshift=-1mm]frame.north west)
					arc[start angle=0,end angle=180,radius=1mm]
					([yshift=-1mm,xshift=1mm]frame.north east)
					arc[start angle=180,end angle=0,radius=1mm];
					\path[left color=yellow!60!black,right color=yellow!60!black,
						middle color=yellow!80!black]
					([xshift=-2mm]frame.north west) -- ([xshift=2mm]frame.north east)
					[rounded corners=1mm]-- ([xshift=1mm,yshift=-1mm]frame.north east)
					-- (frame.south east) -- (frame.south west)
					-- ([xshift=-1mm,yshift=-1mm]frame.north west)
					[sharp corners]-- cycle;
				},interior engine=empty,
		},
	fonttitle=\bfseries,
}{Theorem}


%================================
% Proof
%================================
\newtcbtheorem[number within=section]{Proof}{Proof}{
  enhanced,
	before skip=2mm,after skip=2mm, colbacktitle=violet!15, colback=violet!10, colframe=violet!50!white,boxrule=0.5mm,
	attach boxed title to top left={xshift=1cm, yshift*=1mm-\tcboxedtitleheight}, varwidth boxed title*=-3cm,
	boxed title style={frame code={
					\path[fill=violet]
					([yshift=-1mm,xshift=-1mm]frame.north west)
					arc[start angle=0,end angle=180,radius=1mm]
					([yshift=-1mm,xshift=1mm]frame.north east)
					arc[start angle=180,end angle=0,radius=1mm];
					\path[left color=violet!60!white, right color=violet!60!white,
						middle color=violet!80!white]
					([xshift=-2mm]frame.north west) -- ([xshift=2mm]frame.north east)
					[rounded corners=1mm]-- ([xshift=1mm,yshift=-1mm]frame.north east)
					-- (frame.south east) -- (frame.south west)
					-- ([xshift=-1mm,yshift=-1mm]frame.north west)
					[sharp corners]-- cycle;
				},interior engine=empty,
		},
	fonttitle=\bfseries,
}{Proof}

%================================
% Example
%================================
\newtcbtheorem[number within=section]{Example}{Example}{
  enhanced,
	before skip=2mm,after skip=2mm, colbacktitle=green!15, colback=green!10, colframe=green!80!black,boxrule=0.5mm,
	attach boxed title to top left={xshift=1cm, yshift*=1mm-\tcboxedtitleheight}, varwidth boxed title*=-3cm,
	boxed title style={frame code={
					\path[fill=green]
					([yshift=-1mm,xshift=-1mm]frame.north west)
					arc[start angle=0,end angle=180,radius=1mm]
					([yshift=-1mm,xshift=1mm]frame.north east)
					arc[start angle=180,end angle=0,radius=1mm];
					\path[left color=green!60!black,right color=green!60!black,
						middle color=green!80!black]
					([xshift=-2mm]frame.north west) -- ([xshift=2mm]frame.north east)
					[rounded corners=1mm]-- ([xshift=1mm,yshift=-1mm]frame.north east)
					-- (frame.south east) -- (frame.south west)
					-- ([xshift=-1mm,yshift=-1mm]frame.north west)
					[sharp corners]-- cycle;
				},interior engine=empty,
		},
	fonttitle=\bfseries,
}{Example}




%================================
% Code BOX
%================================
\colorlet{punct}{red!60!black}
\definecolor{background}{HTML}{EEEEEE}
\definecolor{delim}{RGB}{20,105,176}
\colorlet{numb}{magenta!60!black}

\definecolor{mGreen}{rgb}{0,0.6,0}
\definecolor{mGray}{rgb}{0.5,0.5,0.5}
\definecolor{mPurple}{rgb}{0.58,0,0.82}
\definecolor{backgroundColour}{rgb}{255,255,255}

\lstdefinestyle{C}
{
  backgroundcolor=\color{backgroundColour},   
  commentstyle=\color{mGreen},
  keywordstyle=\color{magenta},
  stringstyle=\color{mPurple},
  breakatwhitespace=false,         
  breaklines=true,                 
  captionpos=b,                    
  keepspaces=true,                 
  numbers=left,                    
  showspaces=false,                
  showstringspaces=false,
  showtabs=false,                  
  tabsize=2,
  language=C
}

\lstdefinestyle{Python}
{
  backgroundcolor=\color{backgroundColour},   
  commentstyle=\color{mGreen},
  keywordstyle=\color{magenta},
  stringstyle=\color{mPurple},
  breakatwhitespace=false,         
  breaklines=true,                 
  captionpos=b,                    
  keepspaces=true,                 
  numbers=left,                    
  showspaces=false,                
  showstringspaces=false,
  showtabs=false,                  
  tabsize=2,
  language=Python
}
\lstnewenvironment{Codebox}[2][]
  {
		\lstset{
      numbers=left,
			language=#2,
			basicstyle=\ttfamily,
      frameround=tttt,
      frame = single,
			style=#2,
      caption=#1,
      captionpos=b,
      label=#1,
			nolol
		}
  }
  {
  }

% ================================
% More Subsections
%================================
\titleclass{\subsubsubsection}{straight}[\subsection]
\newcounter{subsubsubsection}[subsubsection]
\renewcommand\thesubsubsubsection{\thesubsubsection.\arabic{subsubsubsection}}
\renewcommand\theparagraph{\thesubsubsubsection.\arabic{paragraph}} % optional; useful if paragraphs are to be numbered
\titleformat{\subsubsubsection}
	{\normalfont\normalsize\bfseries}{\thesubsubsubsection}{1em}{}
\titlespacing*{\subsubsubsection}
{0pt}{3.25ex plus 1ex minus .2ex}{1.5ex plus .2ex}


\titleclass{\subsubsubsubsection}{straight}[\subsection]
\newcounter{subsubsubsubsection}[subsubsubsection]
\renewcommand\thesubsubsubsubsection{\thesubsubsubsection.\arabic{subsubsubsubsection}}
\titleformat{\subsubsubsubsection}
	{\normalfont\normalsize\bfseries}{\thesubsubsubsubsection}{1em}{}
\titlespacing*{\subsubsubsubsection}
{0pt}{3.25ex plus 1ex minus .2ex}{1.5ex plus .2ex}

\makeatletter
\renewcommand\paragraph{\@startsection{paragraph}{5}{\z@}%
	{3.25ex \@plus1ex \@minus.2ex}%
	{-1em}%
	{\normalfont\normalsize\bfseries}}
\renewcommand\subparagraph{\@startsection{subparagraph}{6}{\parindent}%
	{3.25ex \@plus1ex \@minus .2ex}%
	{-1em}%
	{\normalfont\normalsize\bfseries}}
\def\toclevel@subsubsubsection{4}
\def\toclevel@paragraph{5}
\def\toclevel@paragraph{6}
\def\l@subsubsubsection{\@dottedtocline{4}{7em}{4em}}
\def\l@subsubsubsubsection{\@dottedtocline{5}{8em}{5em}}
\def\l@paragraph{\@dottedtocline{5}{10em}{5em}}
\def\l@subparagraph{\@dottedtocline{6}{14em}{6em}}
\makeatother

\setcounter{secnumdepth}{5}
\setcounter{tocdepth}{5}