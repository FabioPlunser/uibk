\newpage

\section{Hallo}
\begin{Lemma}{Testing}{}
	Hallo
\end{Lemma}

\begin{Definition}{Konjunktive und Disjunktive Normalform}{}
	Hallo
\end{Definition}

\begin{Codebox}[Hallo]{C}
	printf("Hallo World");
\end{Codebox}

\begin{Codebox}[Hallo]{Python}
	print("Hallo World");
\end{Codebox}

% \begin{figure}[!htb]
%   \centering
%   \includegraphics[width=\linewidth]{./Test.png}
%   \caption{Test}
%   \label{caption:Test}
% \end{figure}


\begin{Theorem}{Theorem}{}
	proof rules MT, ¬¬i, PBC and LEM are derivable from other (basic) proof rules
\end{Theorem}

\begin{Proof}{Proof}{}
	\begin{logicproof}{2}
    p \rightarrow q & premise \\
    \begin{subproof}
      r \rightarrow p & assumption \\
      \begin{subproof}
        r & assumption \\
        p & $\rightarrow$ e: 2,3 \\
        q & $\rightarrow$ e: 1, 4
      \end{subproof}
        r \rightarrow q & $\rightarrow i: 4,5$
    \end{subproof}
    (r \rightarrow p) \rightarrow (r \rightarrow q) & \, $\rightarrow i$ 3--5
  \end{logicproof}
\end{Proof}


\begin{Example}{}{}
  formula$ ((p \imply q) \imply p) \imply p$ is valid
\end{Example}

\begin{Proof}{Proof}{}
  \begin{karnaugh-map}[4][2][1][q\textsubscript{1}][X][q\textsubscript{0}]
    \manualterms{0,0,0,0,1,D,0,D}
    % \implicant{1}{5}
    \implicant{4}{4}
  \end{karnaugh-map}
\end{Proof}